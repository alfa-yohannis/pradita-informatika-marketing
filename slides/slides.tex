\documentclass[aspectratio=169, table]{beamer}

\usepackage{colortbl}
\usepackage{xcolor}
\usepackage{listings}
\usepackage{tikz}
\usetikzlibrary{positioning, arrows.meta, fit}

\usetheme{Pradita}

\subtitle{Penguatan Konsentrasi
	Keahlian}
\title{\LARGE Dari Teknik Komputer \& Jaringan \\
%\vspace{5pt}
Menuju Karir di Bidang \\
\vspace{4pt}
Teknologi Informasi
}
\date[Serial]{\footnotesize Selasa, 15 Juli 2025 | Tangerang | SMK Islamic Village}
\author{\textbf{Alfa Yohannis}}
\begin{document}

\frame{\titlepage}


\begin{frame}[fragile]
\frametitle{Contents}
\vspace{20pt}
\begin{columns}[t]
	\column{0.5\textwidth}
	\tableofcontents[sections={1-4}]
	
	\column{0.5\textwidth}
	\tableofcontents[sections={5-20}]
\end{columns}
\end{frame}


%\begin{frame}{\hfill}
%	\centering
%	\Huge{\textbf{Are traditional relational databases enough to handle all types of modern business data – or might there be other solutions?}}
%\end{frame}


\section{Siapa Kamu di Dunia TKJ}

\begin{frame}
	\frametitle{Siapa Kamu di Dunia TKJ?}
	\vspace{10pt}
	\begin{itemize}
		\item Siapa di sini yang paling suka instal ulang Windows dan ngoprek jaringan?
		\item Pernahkah kamu:
		\begin{itemize}
			\item Pasang kabel LAN sendiri di rumah?
			\item Bingung kenapa WiFi kantor sekolah "Limited Access"?
			\item Buka CMD cuma buat nge-ping?
		\end{itemize}
		\item Dunia kalian sudah sangat dekat dengan profesi IT modern.
		\item Yuk, mari eksplorasi lebih jauh jalur karier dari TKJ ke dunia digital!
	\end{itemize}
\end{frame}

\section{Apa itu TKJ}
\begin{frame}
	\frametitle{Apa itu TKJ?}
	\vspace{10pt}
	\begin{itemize}
		\item TKJ (Teknik Komputer dan Jaringan) adalah kompetensi dasar dalam dunia IT yang fokus pada:
		\begin{itemize}
			\item Perakitan dan perawatan komputer
			\item Instalasi dan konfigurasi sistem operasi
			\item Desain dan implementasi jaringan komputer (LAN/WAN)
			\item Troubleshooting perangkat dan jaringan
		\end{itemize}
		\item TKJ menjadi fondasi penting untuk berbagai bidang digital: Internet of Things (IoT), Software Development, Cybersecurity, hingga Artificial Intelligence (AI).
		\item Kemampuan teknis yang dimiliki siswa TKJ sangat dibutuhkan dalam dunia kerja dan pendidikan tinggi.
	\end{itemize}
\end{frame}

% Slide 4
\section{Ke Mana Arah Kompetensi TKJ?}
\begin{frame}
	\frametitle{Ke Mana Arah Kompetensi TKJ?}
	\vspace{10pt}
	\begin{itemize}
		\item Lulusan TKJ memiliki bekal praktik yang kuat: instalasi, konfigurasi, dan perawatan sistem.
		\item Dunia digital saat ini membutuhkan peran yang lebih strategis:
		\begin{itemize}
			\item Developer aplikasi dan solusi IoT
			\item Analis keamanan siber
			\item Pengembang sistem cerdas berbasis AI
		\end{itemize}
		\item Untuk naik level, dibutuhkan penguatan di sisi logika, algoritma, dan penguasaan tools modern.
		\item TKJ bukan akhir — tapi awal menuju berbagai profesi digital masa depan.
	\end{itemize}
\end{frame}

% Slide 5
\section{Pintu Masuk Dunia Digital – 3 Bidang Utama}
\begin{frame}
	\frametitle{Pintu Masuk Dunia Digital – 3 Bidang Utama}
	\vspace{10pt}
	\begin{itemize}
		\item \textbf{Cybersecurity}
		\begin{itemize}
			\item Melindungi jaringan dan data dari serangan digital
			\item Cocok bagi siswa TKJ yang senang troubleshooting jaringan
		\end{itemize}
		\item \textbf{IoT dan Software Development}
		\begin{itemize}
			\item Membangun perangkat pintar dan aplikasi berbasis sensor dan konektivitas
			\item Cocok bagi yang suka integrasi hardware dan pemrograman
		\end{itemize}
		\item \textbf{Artificial Intelligence (AI)}
		\begin{itemize}
			\item Menciptakan sistem cerdas untuk analisis data, otomasi, dan machine learning
			\item Cocok bagi yang suka tantangan dan teknologi terbaru
		\end{itemize}
	\end{itemize}
\end{frame}

% Slide 6
\section{Contoh Penerapan Nyata TKJ dalam Dunia Digital}
\begin{frame}
	\frametitle{Penerapan Nyata TKJ dalam Dunia Digital}
	\vspace{10pt}
	\begin{itemize}
		\item \textbf{TKJ → Cybersecurity:}
		\begin{itemize}
			\item Siswa TKJ yang terbiasa konfigurasi router dan firewall bisa berkembang menjadi analis keamanan jaringan.
		\end{itemize}
		\item \textbf{TKJ → IoT dan Software Development:}
		\begin{itemize}
			\item Pengalaman setting perangkat keras dan sistem operasi membuka jalan ke integrasi sensor, mikrokontroler, dan aplikasi IoT.
		\end{itemize}
		\item \textbf{TKJ → AI dan Otomasi:}
		\begin{itemize}
			\item Kemampuan logika dan algoritma dari troubleshooting dapat dikembangkan untuk sistem cerdas berbasis data.
		\end{itemize}
		\item Semua jalur ini dapat dimulai dari keahlian dasar yang sudah dimiliki siswa TKJ.
	\end{itemize}
\end{frame}

% Slide 7
\section{Berapa Potensi Gaji Profesional IT?}
\begin{frame}
	\frametitle{Berapa Potensi Gaji Profesional IT?}
	\vspace{10pt}
	\begin{table}[]
		\centering
		\begin{tabular}{|l|l|}
			\hline
			\textbf{Profesi} & \textbf{Gaji Rata-rata / Bulan} \\
			\hline
			IT Support & Rp 5 – 7 juta \\
			Network Engineer (Junior) & Rp 6 – 10 juta \\
			Software Developer & Rp 10 – 18 juta \\
			Cybersecurity Analyst & Rp 12 – 20 juta \\
			AI / ML Engineer & Rp 18 – 30 juta \\
			\hline
		\end{tabular}
	\end{table}
	\vspace{10pt}
	\begin{itemize}
		\item Gaji dapat lebih tinggi dengan pengalaman, sertifikasi, atau kerja remote.
		\item Sumber: Glassdoor, Jobstreet, LinkedIn, 2025
	\end{itemize}
\end{frame}

\section{Skill Tambahan yang Dibutuhkan}
\begin{frame}
	\frametitle{Skill Tambahan yang Dibutuhkan}
	\vspace{10pt}
	\begin{itemize}
		\item Keterampilan tambahan yang dibutuhkan untuk masuk ke bidang IT profesional:
		\begin{itemize}
			\item \textbf{Artificial Intelligence:} \\
			logika, statistik, pemrograman, pengolahan data, dan dasar machine learning
			\item \textbf{Internet of Things:} \\
			pemrograman perangkat, sensor, komunikasi jaringan, dan protokol ringan
			\item \textbf{Software Development:} \\
			dasar algoritma, desain antarmuka, manajemen kode, dan debugging
			\item \textbf{Cybersecurity:} \\
			pemahaman jaringan, sistem operasi, dan analisis keamanan dasar
		\end{itemize}
		\item Skill ini menjadi fondasi penting untuk studi lanjut dan karier profesional di masa depan.
	\end{itemize}
\end{frame}


% Slide 9
\section{Mengapa Perlu Studi Lanjut?}
\begin{frame}
	\frametitle{Mengapa Perlu Studi Lanjut?}
	\vspace{10pt}
	\begin{itemize}
		\item Dunia industri saat ini tidak hanya membutuhkan pelaksana, tapi juga \textbf{perancang dan pemimpin teknologi}.
		\item Studi lanjut membantu:
		\begin{itemize}
			\item Memahami teori dan praktik secara mendalam
			\item Mendapat akses ke riset, proyek kolaboratif, dan sertifikasi lanjutan
			\item Membangun jaringan profesional dan portofolio akademik
		\end{itemize}
		\item Universitas Pradita menawarkan kurikulum terapan yang selaras dengan kebutuhan industri modern.
		\item Belajar bukan berarti meninggalkan praktik, tapi memperkuat fondasi untuk karier jangka panjang.
	\end{itemize}
\end{frame}

% Slide 10
\section{Belajar di Universitas Pradita}
\begin{frame}
	\frametitle{Belajar di Universitas Pradita}
	\vspace{10pt}
	\begin{itemize}
		\item Universitas Pradita mendukung pembelajaran berbasis praktik dan kolaborasi industri:
		\begin{itemize}
			\item Laboratorium modern untuk AI, IoT, dan Cybersecurity
			\item Fasilitas coworking space, smart classroom, dan perpustakaan digital
			\item Kurikulum adaptif sesuai kebutuhan dunia kerja
			\item Dosen praktisi dari industri teknologi ternama
		\end{itemize}
		\item Suasana kampus yang nyaman dan mendukung pengembangan soft skill dan leadership.
		\item Lokasi strategis di Tangerang, dekat kawasan industri dan bisnis digital.
	\end{itemize}
\end{frame}

% Slide 11
\section{Jalur Masuk dan Beasiswa}
\begin{frame}
	\frametitle{Jalur Masuk dan Beasiswa}
	\vspace{10pt}
	\begin{itemize}
		\item \textbf{Jalur Pendaftaran:}
		\begin{itemize}
			\item Jalur Prestasi Akademik/Nonakademik
			\item Jalur Reguler
			\item Jalur Kemitraan SMK
		\end{itemize}
		\item \textbf{Beasiswa yang Tersedia:}
		\begin{itemize}
			\item Beasiswa Prestasi hingga 100\%
			\item Beasiswa Mitra Sekolah
			\item Beasiswa Bidikmisi/Pradita Peduli
		\end{itemize}
		\item Informasi pendaftaran dan beasiswa dapat diakses melalui:
		\begin{itemize}
			\item Website: \texttt{www.pradita.ac.id}
			\item Instagram: \texttt{@universitaspradita}
			\item QR Code pendaftaran langsung (ditampilkan di slide berikutnya jika ada)
		\end{itemize}
	\end{itemize}
\end{frame}

% Slide 12
\section{Cerita Mahasiswa dari SMK}
\begin{frame}
	\frametitle{Cerita Mahasiswa dari SMK}
	\vspace{10pt}
	\begin{itemize}
		\item Banyak mahasiswa Universitas Pradita berasal dari SMK, termasuk jurusan TKJ dan RPL.
		\item Contoh nyata:
		\begin{itemize}
			\item \textbf{Aldi}, alumni SMK TKJ dari Tangerang, kini aktif di tim riset IoT dan magang di startup teknologi.
			\item \textbf{Nabila}, alumni SMK RPL, sekarang mengembangkan aplikasi Android dan memenangkan lomba UI/UX.
		\end{itemize}
		\item Dukungan dari kampus:
		\begin{itemize}
			\item Proyek nyata dari mitra industri
			\item Bimbingan karier dan program sertifikasi
			\item Kesempatan internasional (webinar, kompetisi, pertukaran)
		\end{itemize}
		\item Bukti bahwa lulusan SMK bisa bersaing dan unggul di dunia teknologi.
	\end{itemize}
\end{frame}

% Slide 13
\section{Ayo Bergabung Bersama Pradita!}
\begin{frame}
	\frametitle{Ayo Bergabung Bersama Pradita!}
	\vspace{10pt}
	\begin{itemize}
		\item Dunia digital membutuhkan talenta seperti kalian: praktis, adaptif, dan siap belajar.
		\item Universitas Pradita siap menjadi jembatan dari keterampilan TKJ menuju profesi masa depan.
		\item \textbf{Akses informasi dan pendaftaran:}
		\begin{itemize}
			\item \texttt{www.pradita.ac.id}
			\item \texttt{@universitaspradita} (IG, TikTok, YouTube)
			\item Scan QR code di slide berikutnya (jika ditambahkan)
		\end{itemize}
		\item \textbf{Kalian adalah bagian dari masa depan teknologi Indonesia.}
		\item Sampai jumpa di Universitas Pradita!
	\end{itemize}
\end{frame}


\end{document}
